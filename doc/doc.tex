\documentclass[12pt, a4paper]{article}

\usepackage[czech]{babel}
\usepackage{lmodern}
\usepackage[utf8]{inputenc}
\usepackage[T1]{fontenc}
\usepackage[pdftex]{graphicx}
\usepackage{amsmath, amssymb}
\usepackage[hidelinks,unicode]{hyperref}
\usepackage{float}
\usepackage{listings}
\usepackage{tikz}
\usepackage{xcolor}
\usepackage{tabularx}
\usepackage[final]{pdfpages}
\usepackage{syntax}
\usepackage{caption}
\usepackage{subcaption}
\usepackage{amsfonts}
\usepackage{siunitx}


\definecolor{mauve}{rgb}{0.58,0,0.82}
\usetikzlibrary{shapes,positioning,matrix,arrows}

\newcommand{\img}[1]{(viz obr. \ref{#1})}

\definecolor{pblue}{rgb}{0.13,0.13,1}
\definecolor{pgreen}{rgb}{0,0.5,0}
\definecolor{pred}{rgb}{0.9,0,0}
\definecolor{pgrey}{rgb}{0.46,0.45,0.48}


\lstdefinestyle{flex}{
    frame=tb,
    aboveskip=3mm,
    belowskip=3mm,
    showstringspaces=false,
    columns=flexible,
    basicstyle={\small\ttfamily},
    numbers=none,
    numberstyle=\tiny\color{black},
    keywordstyle=\color{black},
    commentstyle=\color{black},
    stringstyle=\color{black},
    breaklines=true,
    breakatwhitespace=true,
    tabsize=3
}

\lstset{
    frame=tb,
    language=Python,
    aboveskip=3mm,
    belowskip=3mm,
    showstringspaces=false,
    columns=flexible,
    basicstyle={\small\ttfamily},
    numbers=none,
    numberstyle=\tiny\color{gray},
    keywordstyle=\color{blue},
    commentstyle=\color{pgreen},
    stringstyle=\color{mauve},
    breaklines=true,
    breakatwhitespace=true,
    tabsize=3
}


\let\oldsection\section
\renewcommand\section{\clearpage\oldsection}

\begin{document}
	% this has to be placed here, after document has been created
	% \counterwithout{lstlisting}{chapter}
	\renewcommand{\lstlistingname}{Ukázka kódu}
	\renewcommand{\lstlistlistingname}{Seznam ukázek kódu}
    \begin{titlepage}

        \centering

        \vspace*{\baselineskip}
        \begin{figure}[H]
        \centering
        \includegraphics[width=7cm]{img/fav-logo.jpg}
        \end{figure}

        \vspace*{1\baselineskip}

        \vspace{0.75\baselineskip}

        \vspace{0.5\baselineskip}
        {Semestrální práce z předmětu KIV/PPR\\}
        \vspace{4\baselineskip}
        {\sc Standardní zadání\\}

        {\LARGE\sc Hledání percentilu v souboru\\}

        \vspace{4\baselineskip}

        \vspace{0.5\baselineskip}

        {\sc\Large Stanislav Král \\}
        \vspace{0.5\baselineskip}
        {A20N0091P}

        \vfill

        {\sc Západočeská univerzita v Plzni\\
        Fakulta aplikovaných věd}

    \end{titlepage}

    % TOC
    \tableofcontents
    \pagebreak

\section{Zadání}

Program semestrální práce dostane, jako jeden z parametrů, zadaný souboru, přístupný pouze pro čtení. Bude ho interpretovat jako čísla v plovoucí čárce - 64-bitový double. Program najde číslo na arbitrárně zadaném percentilu, další z parametrů, a vypíše první a poslední pozici v souboru, na které se toto číslo nachází.

Program se bude spouštět následovně: \texttt{pprsolver.exe soubor percentil procesor}

\begin{itemize}
    \item soubor -- cesta k souboru, může být relativní k program.exe, ale i absolutní
    \item percentil -- číslo 1 - 100
    \item procesor -- řetězec určujíící, na jakém procesoru a jak výpočet proběhne
    \begin{itemize}
        \item single -- jednovláknový výpočet na CPU
        \item SMP -- vícevláknový výpočet na CPU
        \item anebo název OpenCL zařízení -- pozor, v systému může být několik OpenCL platforem
    \end{itemize}
\end{itemize}

Součástí programu bude watchdog vlákno, které bude hlídat správnou funkci programu.

Testovaný soubor bude velký několik GB, ale paměť bude omezená na 250 MB. Zařídí validátor. Program musí skončit do 15 minut na iCore7 Skylake.

\section{Analýza}

\section{Závěr}

\end{document}

