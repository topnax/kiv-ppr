\documentclass[xcolor=dvipsnames]{beamer}
%\usepackage[utf8]{inputenc}
%\usepackage{xcolor}
\usepackage[pdftex]{graphicx}
\usepackage{tikz}
\usetikzlibrary{arrows,shapes}
\usepackage{caption}
%\usepackage[utf8]{inputenc}
\usepackage[czech]{babel}
%\usepackage[utf8]{vietnam}
\usepackage{pdfpages}
\usepackage{color}
\usepackage{booktabs}

%%%%%%%%%%%%%%%%%%%%%%%%%%%%%%%%%%%%%%%%%%%%%%%%%%%%%

%%%%%%%%%%%%%%%%%%%%%%%%%%%%%%%%%%%%%%%%%%%%%%%%%%%%%
%\usepackage{lipsum}


%%%%%%%%%%%%%%%%%%%%%%%%%%%%%%%%%%%%%%%%%%%%%%%%%%%%%

\usepackage{pgf}
\usepackage{etex}
\usepackage{tikz,pgfplots}

\tikzstyle{every picture}+=[remember picture]
% By default all math in TikZ nodes are set in inline mode. Change this to
% displaystyle so that we don't get small fractions.
\everymath{\displaystyle}


\usetheme{Madrid}
%\usetheme{Madrid}
%\usecolortheme[named=Maroon]{structure}
%\usecolortheme{crane}
\usefonttheme{professionalfonts}
\useoutertheme{infolines}
\useinnertheme{circles}

\newtheorem*{bem}{Bemerkung}

\usepackage{tikz}


%%%%%%%%%%%%%%%%%%%%%%%%%%%%%%%%%%%%%%%%%%%%%%%%%

%%%%%%%%%%%%%%%%%%%%%%%%%%%%%%%%%%%%%%%%%%%%%%%%%
%\usepackage{listings}
\usepackage{color}

\definecolor{dkgreen}{rgb}{1,0.6,0}
\definecolor{gray}{rgb}{1,1,0}
\definecolor{mauve}{rgb}{0.58,0,0.82}

%%%%%%%%%%%%%%%%%%%%%%%%%%%%%%%%%%%%%%%%%%%%%%%%%

\beamertemplatenavigationsymbolsempty 

\title[závěrečná prezentace SP] %optional
{Hledání percentilu v souboru}

\subtitle{závěrečná prezentace SP z KIV/PPR}

% \title[Západočeská univerzita v Plzni]{\includegraphics[width=\textwidth/4]{img/logo.png}}

\institute[ZČU FAV] % (optional)
{
    Západočeská univerzita v Plzni
    \and
    Fakulta aplikovaných věd
}

\author[Stanislav Král]  % (optional, for multiple authors)
{Stanislav Král \\\tiny{(A20N0091P)}}

%\logo{\includegraphics[height=1.5cm]{img/KIV\_ram\_cerna.pdf}}

\logo{\pgfimage[height=0.5cm]{img/kiv-logo.pdf}}


\begin{document}

\begin{frame}
  \titlepage
\end{frame}

\begin{frame}
\frametitle{Úvod}
	\begin{itemize}
        \item oproti původním diagramům CF a DF pouze minimální nepodstatné změny
        \item navržený algoritmus využívající bitový posun pro výpočet indexu bucketu se osvědčil jako velmi efektivní
        \begin{itemize}
            \item následný návrh efektivní paralelizace nebyl jednoduchý 
        \end{itemize}
  	\end{itemize}
\end{frame}

\begin{frame}
\frametitle{Poznatky z vypracovávání I.}
    \begin{itemize}
        \item v některých případech má použití klíčového slova \texttt{inline} velký (pozitivní) výkonostní dopad
        \begin{itemize}
            \item hlavně pozorováno u klasifikace validity zpracovávaného čísla
            \item vlastní implementace \texttt{std::fpclassify} rychlejší (z učiněných pozorování až o desítky procent)
        \end{itemize}
        \item velikost bufferu při čtení ze souboru výrazně ovlivňuje rychlost celého výpočtu
        \begin{itemize}
            \item optimální velikost se liší u jednotlivých variant výpočtu (\texttt{single}, \texttt{SMP} a OpenCL)
        \end{itemize}
        \item vytvoření třídy pro čtení dat ze souboru s využitím bufferu za účelem sdílení kódu mezi varianty výpočtu vedlo k celkovému zpomalení výpočtů
        \begin{itemize}
            \item z tohoto důvodu není kód čtení ze souboru sdílený a opakuje se
            \item z učiněných pozorování až desítky procent
        \end{itemize}
    \end{itemize}
\end{frame}

\begin{frame} \frametitle{Poznatky z vypracovávání II.}
    \begin{itemize}
        \item SMP původně implementováno pomocí \texttt{parallel\_for}, avšak pomalé, následně nahrazeno implementací pomocí \texttt{flow\_graph}
        \item dokumentace knihovny TBB je dle mého názoru nedostatečná, což mi výrazně ztěžovalo implementaci SMP
        \begin{itemize}
            \item nesrozumitelné chyby během překladu a nedostatek příkladů použití konstrukcí \texttt{flow\_graph}
        \end{itemize}

        \item u optimalizace OpenCL výpočtu hraje podstatnou roli počet zápisu/čtení do/z OpenCL bufferů
        \begin{itemize}
            \item původní implementace výpočtu, kdy bylo OpenCL zařízení využíváno jen k realizaci bitových posunů, trvala jednou tolik času, než sériová implemetace
            \item následná úprava výpočtu tak, aby byl celý histogram sestavován na OpenCL zařízení, výrazně zrychlila tento výpočet
            \item dle naměřených výsledků je pak urychlení oproti sériovému výpočtu 1.32
        \end{itemize}
    \end{itemize}
\end{frame}

\begin{frame}
\frametitle{Poznatky z vypracovávání III.}
    \begin{itemize}
        \item výpočet je mnohem rychlejší na OS Linux než na OS Microsoft Windows
        \begin{itemize}
            \item na Linuxu dochází k jevu, kdy opakovaná spuštění programu nad jedním a tím samým souborem jsou rychlejší
            \item na souboru s obrazem OS Ubuntu se jednalo o zrychlení i několika sekund
        \end{itemize}
        \item během testování na Intel HD zjištěno, že některá volání vytváření OpenCL bufferu a zápisu do něj, která na zařízeních NVidia a AMD fungují v pořádku, na tomto zařízení nefungují zcela správně
        \begin{itemize}
            \item bylo třeba provést drobné úpravy kódu na straně C++
            \item samotný OpenCL kód zůstal stejný
        \end{itemize}
        \item nesprávná práce s OpenCL zařízením má na OS Linux schopnost shodit celý systém
    \end{itemize}
\end{frame}

\begin{frame}
\frametitle{Výsledky}
    \begin{table}
        \begin{center}
            \begin{tabular}{|c||c|c|c|c||c|} 
                 \hline
                 Zařízení & 1. & 2. & 3. & 4. & Průměr[s] \\ 

                 \hline
                 \hline
                  SINGLE &  4.197 & 4.308 & 4.195 & 4.257 & 4.239 \\  
                  SMP & 5.486 & 5.466 & 5.496 & 5.535 & 5.495 \\
                  Ellesmere & 3.294 & 3.069 & 3.374 & 3.105 & 3.211 \\

                 \hline
            \end{tabular}
        \end{center}
        \caption{Časy běhů programu při zpracování souboru s obrazem Ubuntu OS (2.7GiB).}
    \end{table}
\end{frame}

\begin{frame}
\frametitle{Výsledky}
    \begin{figure}[!ht]
    \centering
    \begin{tikzpicture}
    \begin{axis}[
        width=0.66\textwidth,
        ybar=.05cm,
        xlabel={Soubor},
        ylabel={Průměrný čas [s]},
    minor y tick num=2,
        ytick={0,1,2,3,4,5,6,7,8,9,10},
        bar width = 4pt,
        legend style={at={(0.45,0.9)},anchor=west},
        ticklabel style = {font=\tiny},
        symbolic x coords={100MiB,200MiB,500MiB,1000MiB,2500MiB,5000MiB},
        xtick=data]
        \addplot coordinates {
            (100MiB,0.192)
            (200MiB,0.371)
            (500MiB,0.924)
            (1000MiB,1.894)
            (2500MiB,4.67)
            (5000MiB,9.014)
        };

        \addplot coordinates {
            (100MiB,0.248)
            (200MiB,0.458)
            (500MiB,1.074)
            (1000MiB,2.122)
            (2500MiB,5.16)
            (5000MiB,10.24)
        };

        \addplot coordinates {
            (100MiB,0.434)
            (200MiB,0.544)
            (500MiB,0.936)
            (1000MiB,1.614)
            (2500MiB,3.082)
            (5000MiB,5.814)
        };


        \legend {SINGLE, SMP, OpenCL};

    \end{axis}
    \end{tikzpicture}
    \caption{Graf zobrazující data naměřená při spouštění programu pro soubory o různých velikostech.}
\end{figure}
\end{frame}



\begin{frame}
\frametitle{Závěr}
    \begin{itemize}
        \item mým hlavním cílem bylo mít co nejrychlejší výpočet
            \begin{itemize}
                \item avšak bohužel i někdy za cenu čitelnosti kódu
                \item např. u SMP za účelem rychlosti výpočtu zvýšeny paměťové nároky (kopie histogramu pro každé vlákno namísto synchronizovaného přístupu k jednomu jedinému)
            \end{itemize}
        \item bohužel, sériová verze je při použití všech optimalizací při překladu rychlejší než SMP
        \begin{itemize}    
            \item pravěpodobně způsobeno špatným návrhem paralelizace
            \item možnost prozkoumání do budoucna, zdali neexistuje lepší způsob
        \end{itemize}    
    \item na středně výkonném GPU (AMD RX480) výpočet rychlejší než na 12-ti vláknovém procesoru (AMD Ryzen 1600 6C/12T)
    \end{itemize}
\end{frame}

\begin{frame}
\center
Konec prezentace
\end{frame}



\end{document}

